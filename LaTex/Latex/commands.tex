% Abbreviations for 'for example', 'among others'
% etc. In these abbreviations, the space
% between the letters is smaller than
% usual. Therefore it is inserted with '\,'
% instead of a normal space.
% The definitions are terminated with \xspace
% so that, if needed, a space is inserted
% after the macros in the text.
%\newcommand{\zB}{z.\,B.\xspace} % for example (z. B.)
%\newcommand{\ua}{u.\,a.\xspace} % among others (u. a.)
%\newcommand{\uU}{u.\,U.\xspace} % under certain circumstances (u. U.)

% Short forms for existing, long LaTeX commands:
\newcommand{\tb}{\textbackslash}

% Commands for reference (ref) commands:
\newcommand{\chapterref}[1]{Chapter~\ref{#1}}
\newcommand{\sectionref}[1]{Section~\ref{#1}}
\newcommand{\subsectionref}[1]{Subsection~\ref{#1}}
\newcommand{\equationref}[1]{Eq.~\ref{#1}}
\newcommand{\figureref}[1]{Figure~\ref{#1}}
\newcommand{\tableref}[1]{Table~\ref{#1}}
\newcommand{\appref}[1]{Appendix~\ref{#1}}

% Commands for mathematical constructs:

% Absolute value bars with correct size around any object:
\newcommand{\abs}[1]{\left|#1\right|}

% Derivatives with upright printed differential operator:
\newcommand{\deriv}[2]{\frac{\mathrm{d} #1}{\mathrm{d} #2}}

% Upright printed Euler's number and imaginary unit:
\newcommand{\euler}{\mathrm{e}}
\newcommand{\imag}{\mathrm{i}}

% Expected value
\DeclareMathOperator{\EX}{\mathbb{E}}% expected value
\newcommand*\mean[1]{\overline{#1}}

% Command for align environment to show transformation steps
\newcommand{\sh}[2]{&& \quad \vert\ \text{#1} #2\\}

\renewcommand{\thesection}{\Alph{section}}
